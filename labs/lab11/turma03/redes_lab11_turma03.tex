\documentclass{article}
\usepackage{xcolor}
\usepackage{hyperref}

\usepackage{enumitem}
\usepackage{wasysym}
\usepackage{geometry}
% Custom command for incorrect options
\newcommand{\incorrectOption}{\item[\textbf{\Circle}]}
% Custom command for correct option
\newcommand{\correctOption}{\item[\textbf{\CIRCLE}]}

\definecolor{darkgreen}{rgb}{0,0.8,0.3}
\definecolor{darkblue}{rgb}{0.00,0.00,0.55}


\begin{document}

\begin{center}
\rule{\textwidth}{.0075in} \\
\rule[3mm]{\textwidth}{.0075in}\\

Universidade do Algarve\hfill \textbf{Cross-Site Scripting (XSS)} \hfill 2024\\[3ex]

{\Large\bf Lab \#11} \\[3ex]

\begin{tabular}{l}
\textcolor{blue}{PL3} \\ \hfill 
\end{tabular}
\begin{tabular}{l}
\textcolor{blue}{Email:} \\ \hfill 
\end{tabular}
\begin{tabular}{l}
\textcolor{blue}{Data de entrega: Dezembro-06-2024 pelas 16h30} \\ \hfill 
\end{tabular}\\

\rule{\textwidth}{.0075in} \\
\rule[3mm]{\textwidth}{.0075in} \\
\end{center}

\section*{Cross-Site Scripting Attacks}
You should answer two of the next questions depending on the last digit of your student
number. Hence, if your student number is \texttt{a65873} you should answer
questions 3 and 13. Your should turn-in your answers in Tutoria.

\bigskip 
\bigskip

{\large
  \noindent \textbf{Questions refer to \texttt{slides$\_$13} in Tutoria}. 
}
\bigskip
\bigskip

\begin{enumerate}
  %
\item[0.] Slide 19. Why does it meant ``\textbf{to leak data
    cross-origin}''? \textbf{Explain}. 

\item  Slide 37. Which \textbf{value} would in this case be? Why
  would the browser assume that it is talking to Alice?
  \textbf{Amplify on what it was discussed in class}.
%
\item Slide 13. Explain the message at the botton of the slide: ``Same-origin policy: it
  prevents script access from \texttt{hacker.net} to the \texttt{victim.org}’s
  DOM''. What does this message mean? 
  \textbf{Amplify on what it was discussed in class}.
  %
\item Slides 1 to 22. What are the final 3 conclusions regarding the
  Web Security that were given at the end of the class?
  \textbf{Explain them}. 
  %
\item Slide 6. According to this
  web-site~\href{https://owasp.org/www-project-developer-guide/draft/foundations/security_principles}{Link},
  ``the least common security principle disallows the sharing of
  mechanisms that are common to more than one user or process if the
  users or processes are at different levels of privilege''. How does
  the least common security principle relate to the same-policy origin
  principle? \textbf{Amplify on what it was discussed in class}.
  %
\item Slide 8. Why do we mention here the term ``complete mediation''? \textbf{Amplify on what it
  was discussed in class}. 
  %
\item Slide 19. In which of these categories ``cookie theft'' can be placed? \textbf{Amplify on
  what it was discussed in class}. 
%   %
\item Slide 16. What is meant by ``the script  tag is not subject to the Same-Origin
  Policy''? What are the consequences of that? 
  \textbf{Amplify on what it was discussed in class}.
  %
\item Slide 11. Explain the 3 different cases shown in the Slide. Why do these sites
  have different origins?  \textbf{Amplify on what it was discussed in class}. 
  %
\item Slide 4. How pervasive are web attacks? \textbf{Amplify on what it
  was discussed in class}. 
  %
\item Slide 5. How pervasive are XSS attacks? \textbf{Amplify on what it
  was discussed in class}.  
% 
\item Slide 6. According to this
  web-site~\href{https://owasp.org/www-project-developer-guide/draft/foundations/security_principles}{Link},
  ``the least common security principle disallows the sharing of
  mechanisms that are common to more than one user or process if the
  users or processes are at different levels of privilege''. How does
  the \textbf{least common} security principle relate to the same-policy origin
  principle? \textbf{Amplify on what it was discussed in class}.
%
\item Slide 6. According to this
  web-site~\href{https://owasp.org/www-project-developer-guide/draft/foundations/security_principles}{Link},
  ``the least common security principle disallows the sharing of
  mechanisms that are common to more than one user or process if the
  users or processes are at different levels of privilege''. How does
  the \textbf{least common} security principle relate to the separation model
  of Google Chrome? \textbf{Amplify on what it was discussed in class}.
%
\item Slide 8. According to this
  web-site~\href{https://owasp.org/www-project-developer-guide/draft/foundations/security_principles}{Link},
  the \textbf{complete mediation} security principle ensures that
  authority is not circumvented in subsequent requests of an object by
  a subject, by checking for authorization (rights and privileges)
  upon every request for the object. Why is \textbf{complete
    mediation} mentioned in Slide 8? \textbf{Amplify on what it was discussed in class}.
  %
\item Slide 20. Explain the security breach that occurs 
  here.  \textbf{Amplify on what it was discussed in class}.
  %
\item Slide 21. Explain the sanitisation process that must occur here.
  \textbf{Amplify on what it was discussed in class}.
  %
\item Slide 37. If an attacker steals Alice's cookie, how would this
  attacker impersonate Alice?  \textbf{Amplify on what it was
    discussed in class}.
  %
\item Slide 41. The slide shows 4 items in red colour. Explain how
  these 4 items impact negatively the security of web applications.
  %
\item Slide 42. Explain the security issue that second paragraph
  suggests.  
  %
\item Slide 7. What is the trade-off between \textbf{flexibility}
  and \textbf{security}? \textbf{Explain}.
\end{enumerate}  


\bigskip

\section*{Critérios de Avaliação}

\begin{tabular}{|c|p{2cm}|} \hline
  \textbf{Value} & \textbf{Description} \\ \hline
  5 & Excellent \\ \hline
  4 & Good  \\ \hline 
  3 & Average  \\ \hline 
  2 & Bad \\ \hline 
  1 & really? \\ \hline 
\end{tabular}

\bigskip 

\bigskip 

\bigskip

\textbf{Your answers go here}.


\end{document}
