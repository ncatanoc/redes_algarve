\documentclass{article}
\usepackage{xcolor}
\usepackage{hyperref}

\usepackage{enumitem}
\usepackage{wasysym}
\usepackage{geometry}
% Custom command for incorrect options
\newcommand{\incorrectOption}{\item[\textbf{\Circle}]}
% Custom command for correct option
\newcommand{\correctOption}{\item[\textbf{\CIRCLE}]}

\definecolor{darkgreen}{rgb}{0,0.8,0.3}
\definecolor{darkblue}{rgb}{0.00,0.00,0.55}


\begin{document}

\begin{center}
\rule{\textwidth}{.0075in} \\
\rule[3mm]{\textwidth}{.0075in}\\

Universidade do Algarve\hfill \textbf{Cross-Site Scripting} \hfill 2024\\[3ex]

{\Large\bf Lab \#11} \\[3ex]

\begin{tabular}{l}
\textcolor{blue}{Email} \\ \hfill 
\end{tabular}
\begin{tabular}{l}
\textcolor{blue}{Data de entrega: Dic-03-2024 pelas 16h30} \\ \hfill 
\end{tabular}\\

\rule{\textwidth}{.0075in} \\
\rule[3mm]{\textwidth}{.0075in} \\
\end{center}

\section*{Injection Attacks}
You should answer two of the next questions depending on the last 2 digits of your student
number.

\bigskip

\noindent
\textbf{First question.} Answer the question that matches the last
digit of your student number, e.g., if your student number is
\texttt{a65873} you should answer question 3.

\bigskip

\noindent
\textbf{Second question.} Answer the question that matches the
second-last (``penúltimo'') digit of your student number, e.g. if your
student number is \texttt{a65873} you should answer question
7. However, if the last two digits of your student number are the
same, e.g. as in \texttt{a71566}, then you should take the sum of the last
two digits module 10, e.g., 2 = (6+6) module 10.

\bigskip

{\large
  \noindent \textbf{These questions were discussed in class. The questions
    refer to \texttt{slides$\_$13} in Tutoria. Turn in your answers
    through Tutoria}.  }

\bigskip
\bigskip

\begin{enumerate}
  %
\item[0.] Slide 13. Explain the message at the botton of the slide: ``Same-origin policy: it
  prevents script access from \texttt{hacker.net} to the \texttt{victim.org}’s
  DOM''. What does this message mean? 
  \textbf{Amplify on what it was discussed in class}.
  %
\item Slide 20, 21. What is the line ``alert(JSON.stringify(confidential\_keys[0]))''
  doing? \textbf{Amplify on what it was discussed in class}.
  %
\item Slides 1 to 22. What are the final 3 conclusions regarding the
  Web Security that were given at the end of the class?
  \textbf{Explain them}. 
  %
\item Slide 6. Why do we mention here the ``least common mechanism''? \textbf{Amplify on what it
  was discussed in class}.
  %
\item Slide 8. Why do we mention here the term ``complete mediation''? \textbf{Amplify on what it
  was discussed in class}. 
  %
\item Slide 19. In which of these categories ``cookie theft'' can be placed? \textbf{Amplify on
  what it was discussed in class}. 
%   %
\item Slide 16. What is meant by ``the script  tag is not subject to the Same-Origin
  Policy''? What are the consequences of that? 
  \textbf{Amplify on what it was discussed in class}.
  %
\item Slide 11. Explain the 3 different cases shown in the Slide. Why do these sites
  have different origins?  \textbf{Amplify on what it was discussed in class}. 
  %
\item Slide 4. What does this graphic suggest? \textbf{Amplify on what it
  was discussed in class}. 
  %
\item Slide 5. What does this graphic suggest? \textbf{Amplify on what it
  was discussed in class}.
\end{enumerate}


\bigskip

\section*{Critérios de Avaliação}

\begin{tabular}{|c|p{2cm}|} \hline
  \textbf{Value} & \textbf{Description} \\ \hline
  5 & Excellent \\ \hline
  4 & Good  \\ \hline 
  3 & Average  \\ \hline 
  2 & Bad \\ \hline 
  1 & really? \\ \hline 
\end{tabular}

\bigskip 

\bigskip 

\bigskip

\textbf{Your answers go here}.


\end{document}
