\documentclass{article} % set_input_method
\usepackage{xcolor}
\usepackage{hyperref}

\usepackage{enumitem}
\usepackage{wasysym}
\usepackage{geometry}
% Custom command for incorrect options
\newcommand{\incorrectOption}{\textbf{\Circle}}
% Custom command for correct option
\newcommand{\correctOption}{\textbf{\CIRCLE}}

\definecolor{darkgreen}{rgb}{0,0.8,0.3}
\definecolor{darkblue}{rgb}{0.00,0.00,0.55}

\begin{document}

\begin{center}
\rule{\textwidth}{.0075in} \\
\rule[3mm]{\textwidth}{.0075in}\\

Universidade do Algarve\hfill Basics on Security \hfill 2024\\[3ex]

{\Large\bf Lab \#5} \\[3ex]

 \textcolor{blue}{Nome} \hfill  
 \textcolor{blue}{PL1} \hfill
 \textcolor{blue}{Número de Aluno}\\

\rule{\textwidth}{.0075in} \\
\rule[3mm]{\textwidth}{.0075in} \\
\end{center}


\begin{center}
\rule{\textwidth}{0.4pt}
\textbf{Completely fill the circles as shown: \Circle \Circle \CIRCLE \Circle}
\rule{\textwidth}{0.4pt}
\end{center}


\bigskip
\bigskip

{\Large
\begin{tabular}{|c|c|} \hline
\textbf{Rúbrica de Avaliação}  & \textbf{Pontos}\\ \hline
Correct answer selected & 2\\  \hline
Incorrect answer selected & -1\\  \hline
\end{tabular}
}

\bigskip
\bigskip

\begin{enumerate}[leftmargin=*, label=\textbf{Q\arabic*}, itemsep=-5pt]
\item You listen to a presentation about a new network protocol for
  online banking. After the talk, there is a lot of discussion going
  on. \textbf{Check all the remarks} that are relevant under the \textbf{Dolev-Yao} model.  
  \begin{enumerate}[itemsep=-3.5pt, label={}]
   \item[\incorrectOption] The bank runs Windows on their servers. This will
    be insecure.
   \item[\incorrectOption] It looks nice, the NSA (National Security Agency)
    will break the encryption function and use this to spy on us.
    \item[\incorrectOption] What happens if someone breaks into the bank's
    data center? They should use a blockchain instead!
    \item[\incorrectOption] I don't think they properly protect against
    transaction replay.
   \end{enumerate}   
  %
\item For each network capability, which goal is {\bf{}directly}
  violated? {\bf{}Only check one goal} - the most directly violated
  one per capability.

  \begin{tabular}{|c|c|c|c|} \hline
     \textbf{Attacker Capability} & \textbf{Confidentiality} & \textbf{Integrity} & \textbf{Availability}\\ \hline
     Observe packets & \incorrectOption & \incorrectOption & \incorrectOption\\ \hline
     Modify packets & \incorrectOption & \incorrectOption & \incorrectOption\\ \hline
     Drop packets & \incorrectOption & \incorrectOption & \incorrectOption\\ \hline
     Delay packets & \incorrectOption & \incorrectOption & \incorrectOption\\ \hline
     Forge packets & \incorrectOption & \incorrectOption & \incorrectOption\\ \hline
     Replay packets & \incorrectOption & \incorrectOption & \incorrectOption\\\hline
  \end{tabular}  \\
%
  \item \textbf{Check all statements that are true about BGPs}.
  \begin{enumerate}[itemsep=-3.5pt, label={}]
  \item[\incorrectOption]  The confidentiality of our communications
    can be asserted by physically protecting all fiber cables (in the
    world) on the default path.  
  \item[\incorrectOption] Securing BGP communications \textbf{today}
    is mostly an \textbf{afterthought} due to the fact that threat
    models have changed from the old Arpanet to the modern Internet.
  \item[\incorrectOption]  When leaving a spouse (home), one should
    reconfigure their BGP routes to protect her against stalking.
  \end{enumerate}
%
  \item Your device has joined a new network that uses DHCP to assign
    you an IP address. What is the first thing that happens to get
    your new IP address? \textbf{Check all the statements that are true}.
  \begin{enumerate}[itemsep=-3.5pt, label={}]
  \item[\incorrectOption]  Your device asks for an IP address
    directly fron the DHCP server. 
  \item[\incorrectOption]  Your device broadcasts a DHCP request to
    all the clients on the network. 
  \item[\incorrectOption] The DHCP server sends an announcement and
    your client responds. 
  \item[\incorrectOption] Santa gets your request, checks his list and
    grants an address depending on whether your devices has been bad
    or good. 
  \end{enumerate}    
%
  \item Which statements are true about DHCP spoofing? \textbf{Check all the
    statements that are true}.
  \begin{enumerate}[itemsep=-3.5pt, label={}]
    \item[\incorrectOption]  a client fools the DHCP server into
      giving it an IP address when it is unauthorised.
%
    \item[\incorrectOption]  an imposter DHCP server fools the client
      into thinking it is the real DHCP server.
  \end{enumerate}
  
\end{enumerate}



\end{document}
