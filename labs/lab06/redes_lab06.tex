\documentclass{article} % set_input_method
\usepackage{xcolor}
\usepackage{hyperref}

\usepackage{enumitem}
\usepackage{wasysym}
\usepackage{geometry}
% Custom command for incorrect options
\newcommand{\incorrectOption}{\textbf{\Circle}}
% Custom command for correct option
\newcommand{\correctOption}{\textbf{\CIRCLE}}

\definecolor{darkgreen}{rgb}{0,0.8,0.3}
\definecolor{darkblue}{rgb}{0.00,0.00,0.55}

\begin{document}

\begin{center}
\rule{\textwidth}{.0075in} \\
\rule[3mm]{\textwidth}{.0075in}\\

Universidade do Algarve\hfill Anti IP Spoofing \hfill 2024\\[3ex]

{\Large\bf Lab \#6} \\[3ex]

\begin{tabular}{l}
\textcolor{blue}{Nome 1} \\ \hfill 
\textcolor{blue}{Nome 2} \hfill
\end{tabular}
 ~~~~~\textbf{Due Date}:~~~Friday, 18th of October, 2024 \hfill 
\begin{tabular}{l}
\textcolor{blue}{Número de Aluno 1} \\ \hfill 
\textcolor{blue}{Número de Aluno 2} \hfill
\end{tabular}\\

\rule{\textwidth}{.0075in} \\
\rule[3mm]{\textwidth}{.0075in} \\
\end{center}


Take module 5 of MANRS tutorials, which is reachable from 
\href{https://www.manrs.org/resources/tutorials/anti-spoofing/}{https://\-www.manrs.org/\-resources/\-tutorials/\-anti-\-spoofing/}
and answer the next questions. Additionally, you are advised to take the Anti-Spoofing quiz you
will be presented at the end of the tutorial.

For this Lab, you should work in groups of two students. Make sure you edit this file to include
both of your names and the student numbers.


\begin{enumerate}
\item Discuss and explain a few techniques by which IP spoofing can be prevented (for instance,
  Access Control Lists and uRPF)? What is each technique about? What do the techniques differ from
  each other? What is each technique good for?
  %
\item Where should anti-spoofing measures be applied
  (for instance points or devices in your network topology)?
  %
\item How do you verify that your anti-spoofing solution works?
\end{enumerate}  
\end{document}



