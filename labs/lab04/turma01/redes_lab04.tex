\documentclass{article}
\usepackage{hyperref}

\usepackage{xcolor,enumerate,hyperref}

\definecolor{darkgreen}{rgb}{0,0.8,0.3}
\definecolor{darkblue}{rgb}{0.00,0.00,0.55}

\begin{document}

\begin{center}
\rule{\textwidth}{.0075in} \\
\rule[3mm]{\textwidth}{.0075in}\\

Universidade do Algarve\hfill Myths and truths about IPv6 \hfill 2024\\[3ex]

{\Large\bf Lab \#4} \\[3ex]

 \textcolor{blue}{Nome} \hfill  
 \textcolor{blue}{PL1} \hfill
 \textcolor{blue}{Número de Aluno}\\

\rule{\textwidth}{.0075in} \\
\rule[3mm]{\textwidth}{.0075in} \\
\end{center}

\bigskip

\noindent
IPv6 is still widely misunderstood among engineers across the
globe. From a security standpoint, IPv4 and IPv6 are very
similar. Let's try to dispel of the most common security myth
regarding IPv6. You must decide whether the following myth is true or
false. Justify your answer (\textcolor{red}{in Portuguese}) based on
the link that's provided below. Explain in about 10 to 20 lines of
text if the following is a myth or not. Use your own words! Do not
copy the text from the link and copy it as it were your own answer.

\noindent
\paragraph{Myth:} A Man-in-the-Middle attack is impossible with IPv6.
\href{https://www.eweek.com/security/attackers-can-use-ipv6-to-launch-man-in-the-middle-attacks}{\textcolor{red}{Click
    this link}}.\\

\bigskip

\noindent

{\LARGE
\noindent
Your answer goes here. \\

\medskip

\noindent
And, here.
}

\end{document}
